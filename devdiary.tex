\chapter{Entwicklertagebuch}

\section{Markus}

\subsection{12. Dezember 2012, 1. Tag}

\begin{itemize}
	\item Arduino Duemilanove
	\item AtmelStudio 6
	\item Flashen mit dem AVRISP MkII-Kompatiblen
	\item LED blinken lassen
	\item Button mit externem Interrupt
	\item USB-Host zerschossen mit Kurzschluss
	\item Externer Pin-Interrupt auf Flankenwechsel für LED
\end{itemize}

\subsection{13. Dezember 2012, 2. Tag}

\begin{itemize}
	\item USART, Bluetooth
	\item Ringpuffer für Empfang und Senden
	\item Asynchroner Empfang, UDRE-Interrupt
\end{itemize}

\subsection{14. Dezember 2012, 3. Tag}

\begin{itemize}
	\item USART-Kommunikation abstrahiert
	\item Sendepuffer viel zu klein für kürzere Texte
	\item Blockieren, wenn Puffer voll
	\item Kommandoparser begonnen, funktioniert
	\item Erkennung von \texttt{\%C} und \texttt{\%D} mit überflüssigen Daten (hardcoded state machine für Parsing)
\end{itemize}

Special guest:

\begin{itemize}
	\item Roving Networks RN-41, BlueSMiRF Gold - PIO 2 auf Breakout, Kabel zur Hardware-Erkennung von Verbindungen
\end{itemize}

\subsection{15. Dezember 2012, 4. Tag}

\begin{itemize}
	\item Kommandodekoder überarbeitet, Fehlertolerant gestaltet
	\item Erweiterung der State Machine um Befehlsdefinitionen
	\item Erkennung von \texttt{\%CONNECT} und \texttt{\%DISCONNECT}
\end{itemize}