\todo{static thrust vs. jet velocity (pitch speed?)}

\section{Thrust}

\begin{quote}
So basically, thrust is air pushing against air. There are two Components of Thrust; The "Size" (power) and "Velocity" (speed) of that push. A 8.5 inch has twice the area as a 6 inch pipe. A 8.5v inch pipe with air coming out of it at 40mph will produce the same thrust as a 6 inch pipe with a air coming out of it at 80mph. A 7.25 inch pipe with a air coming out of it at 60mph will be a balance between the two extremes. All three pipes are moving exactly the same amount of air and have the same amount of thrust. If like motors (with the same amp draw) produce the thrust in each pipe; the 8.5 inch pipe’s thrust will be powerful and slow, the 6 inch pipe’s thrust will be weak and fast, and the 7.25 inch pipe will have a balance of power and speed.
\end{quote}
\todo{http://www.rcpowers.com/community/threads/the-components-of-thrust-how-to-choose-a-propeller-tutorial.15443/}

\begin{quote}
Ein Senkrechtstarter kann nur dann senkrecht abheben, wenn die Schubkraft größer ist als die Gewichtskraft des Flugzeugs, siehe auch Schub-Gewicht-Verhältnis. Bei einem 17 Tonnen schweren Hawker Siddeley Harrier z. B. reichen die 200 kN aus seinem Triebwerk aus, um ihn vertikal zu beschleunigen. Bei Starrflügelflugzeugen muss der Schub nur einen Bruchteil des Eigengewichts betragen, da der Flügel den anderen Teil des Eigengewichtes „trägt“.
\end{quote}
\todo{http://de.wikipedia.org/wiki/Schub}

\begin{figure}
	% from tdycyl at http://tex.stackexchange.com/questions/63370/drawing-3d-cylinder
\newcommand{\rotorcylinder}[5]{% origin x, origin y, origin z, radius, height
    \path (1,0,0);
    \pgfgetlastxy{\cylxx}{\cylxy}
    \path (0,1,0);
    \pgfgetlastxy{\cylyx}{\cylyy}
    \path (0,0,1);
    \pgfgetlastxy{\cylzx}{\cylzy}
    \pgfmathsetmacro{\cylt}{(\cylzy * \cylyx - \cylzx * \cylyy)/ (\cylzy * \cylxx - \cylzx * \cylxy)}
    \pgfmathsetmacro{\ang}{atan(\cylt)}
    \pgfmathsetmacro{\ct}{1/sqrt(1 + (\cylt)^2)}
    \pgfmathsetmacro{\st}{\cylt * \ct}
    \filldraw[fill=white, draw opacity=0.5, fill opacity=0] (#4*\ct+#1,#4*\st+#2,#3) -- ++(0,0,#5) arc[start angle=\ang,delta angle=-180,radius=#4] -- ++(0,0,-#5) arc[start angle=\ang+180,delta angle=-180,radius=#4];
    \filldraw[fill=white, draw opacity=0.5, fill opacity=0] (#1,#2,#3+#5) circle[radius=#4];
}

\centering
\begin{minipage}[b]{0.4\textwidth}

	\tdplotsetmaincoords{75}{145}
	\begin{tikzpicture}[tdplot_main_coords]

		\draw [black, dotted, fill=black, fill opacity=0.1] (0,0) circle (3);
		\pic{rotor};

		%\rotorcylinder{0}{0}{-2}{3}{4};
		\draw [dotted, >->] (0,0,2) -- (0,0,-2);
		
		\def\radiusA{(tanh(2)+3)}
		\def\radiusB{(tanh(2)+3)}
		\def\radiusC{(tanh(1)+3)}
		\def\radiusD{(tanh(-1)+3)}
		\def\radiusE{(tanh(-2)+3)}
		\def\radiusF{(tanh(-2)+3)}
			
		\draw [draw opacity=0.3] (0,0,2) circle (\radiusA);
		\draw [draw opacity=0.3]  (0,0,1) circle (\radiusB);
		\draw [draw opacity=0.3]  (0,0,0.35) circle (\radiusC);
		\draw [draw opacity=0.3]  (0,0,-0.35) circle (\radiusD);
		\draw [draw opacity=0.3]  (0,0,-1) circle (\radiusE);
		\draw [draw opacity=0.3]  (0,0,-2) circle (\radiusF);
		
		\foreach \t in {-180,-160,...,160}{
			\def\x{cos(\t)}
			\def\y{sin(\t)}
			
			\draw [gray,draw opacity=0.2] plot [smooth,tension=0.3] coordinates {
					({\x*\radiusA},{\y*\radiusA},2)
					({\x*\radiusB},{\y*\radiusB},1) 
					({\x*\radiusC},{\y*\radiusC},0.35)
					({\x*\radiusD},{\y*\radiusD},-0.35)
					({\x*\radiusE},{\y*\radiusE},-1)
					({\x*\radiusE},{\y*\radiusE},-2)
					};
		}
			
	\end{tikzpicture}

\end{minipage}
\hfill
\begin{minipage}[b]{0.4\textwidth}

\begin{tikzpicture}
	
	\draw [<->] (0,2) node (yaxis) [above] {$+h$} -- (0,0) -- (2,0) node (p_inf) [above right,cyan] {$p_{\infty}$} -- (4,0) node (xaxis) [above right,cyan] {$\Delta p$} node (xaxis) [below right,red] {$\Delta v$};
	\draw [->] (0,0) -- (0,-2) node (xaxis) [below] {$-h$};
	\draw [cyan,dotted] (2,2) -- (2,-2);
	
	\draw [cyan] plot [smooth,tension=1] coordinates {(2,2) (1.5,0.5) (0,0)};
	\draw [cyan] plot [smooth,tension=1] coordinates {(2,-2) (2.5,-0.5) (4,0)};
	
	\draw [red,dashed] plot [smooth,tension=1] coordinates {(3.75,-2) (3,-0.5) (1,0.5) (0.25,2)};
	
\end{tikzpicture}
\end{minipage}
	\caption[Schematic of the rotor actuator disc concept]
			{Schematic of the rotor actuator disc concept: Stream tube geometry (left),
			 change in air velocity ($\Delta v$, dotted red) and
			 air pressure relative to 
			 atmospheric pressure $\rho_\infty$ ($\Delta \rho$, solid cyan).}
	\label{fig:actuator_disc}
\end{figure}

\section{Static thrust}

\todo{two different types of thrust, static and ...}

Static thrust\index{thrust!static} % Standschub
is the thrust force applied by an unmoved (\text{static}) system to its environment.

In case of a propeller based system, static thrust is the amount of thrust 
produced by the propeller when it is stationary in relation to the ground.

Viewed differently, it describes the force required to keep a system static if its 
thrusters operate with full power. \todo{what is it good for}

\todo{static thrust is also what makes a helicopter hover; the static thrust must at least overpower the gravitational force}

\todo{thrust-to-weight ratio}
\todo{https://quadcopterproject.wordpress.com/static-thrust-calculation/}
\todo{http://electricrcaircraftguy.blogspot.de/2013/09/propeller-static-dynamic-thrust-equation.html}

\bigbreak

Aus der Drehzahl 
$N \withunit{\per\minute}$ 
des Motors, sowie des Radius 
$r \withunit{\metre}$ 
der Luftschraube lässt sich die Winkelgeschwindigkeit 
$\omega_R \withunit{\radian\per\second}$, 
sowie die Blattspitzen- bzw. Umfangsgeschwindigkeit 
$U \withunit{\metre\per\second}$ 
ermitteln:

\begin{align}
	\omega_R &= 2\pi \cdot \frac{N}{60 \si{\per\second}} \approx 0.105 \cdot N \\
	U &= \omega_R \cdot r
\end{align}

\todo{Quellen aus \url{http://www.rc-network.de/magazin/artikel_02/art_02-0037/Standschub.pdf}}
\bigbreak

Das die Luftschraube treibende Drehmoment $M$ $\withunit{\newton\meter}$ ist mit der Antriebs-/Wellenleistung $P_W \withunit{\watt}$ des Motors über den Zusammenhang

\begin{align}
	R_W = M \cdot \omega_R
\end{align}

gekoppelt.

\bigbreak \todo{Zusammenhang...?}

Die vom Motor auf den Propeller übertragene Schubleistung $P_S$ lässt sich über die Formel

\begin{align}
	P_S = K_p \cdot \left(\frac{N}{1000}\right)^{K_t}
\end{align}

ermitteln. $K_p$ entspricht hierbei der Propellerkonstante, $K_t$ einer Leistungskonstante \todo{Quelle finden, \url{http://quadcopterproject.wordpress.com/static-thrust-calculation/} ist da sehr vage} und $N \withunit{\per\minute}$ der bekannten Drehzahl des Motors.

\bigbreak

Alternativ kann unter Vorgabe des Durchmessers $D \withunit{\inch}$ der Luftschraube und ihrer Steigung $S \withunit{\inch}$ die Leistung bei Propellerkonstanten $1.1 \leq K_p \leq 1.3$ abgeschätzt werden\todo{Quelle auf \url{http://corsair.flugmodellbau.de/files/proggies/PROPSC~1.TXT}}:

\begin{align}
	P_S = K_p \cdot \left(\frac{D}{12 \si{\inch\per\foot}}\right)^4 \cdot \frac{S}{12 \sfrac{in}{ft}} \cdot \left(\frac{N}{1000}\right)^3
\end{align}

Für eine $14\times 10$-Luftschraube mit den Werten $K_p = 1.118$ und $K_t = 3.20$ bei $N = 12000 \si{\per\minute}$ \todo{Beispiel für APC Electric $14\times 10$ auf \url{http://aircraft-world.com/prod_datasheets/hp/emeter/hp-propconstants.htm}} ergäbe sich folglich:

\begin{align}
	P_{K_p,K_t} &= 1.118 \cdot 12^{3.20} \\ &= 3175.6 \si{\watt} \notag\\
	P_{approx} &= 1.118 \cdot \left(\frac{14}{12}\right)^4 \left(\frac{10}{12}\right) \cdot \left(\frac{12000}{1000}\right)^3 \\ &= 2982.5 \si{\watt} \notag
\end{align}

Für kleinere Werte von $K_p$ weichen die Ergebnisse jedoch stark voneinander ab, so dass das Beispiel $K_p = 0.187$ und $K_t = 3.30$ bei $N = 12000 \si{\per\minute}$ \todo{APC SLOWFLY $10\times 4.7$} nicht funktioniert:

\begin{align}
	P_{K_p,K_t} &= 0.187 \cdot 12^{3.30} \\ &= 681 \si{\watt} \notag\\
	P_{approx} &= 0.187 \cdot \left(\frac{14}{12}\right)^4 \left(\frac{10}{12}\right) \cdot \left(\frac{12000}{1000}\right)^3 \\ &= 61 \si{\watt} \notag
\end{align}

\bigbreak

Gemäß \cite{standschub} kann der Schub einer Luftschraube wie folgt ermittelt werden:

\begin{align}
	K_p &= 0.0856 \cdot \frac{H}{D} - 0.0091 \\
	P_S &= K_p \cdot \rho \left(\frac{N}{60}\right)^3 \cdot D^5 \\
	F_S &= 0.67 \sqrt[3]{\frac{\rho}{2}\cdot\pi \cdot D^2 \cdot P_S^2}
\end{align}

wobei $H,D \withunit{\centi\meter}$ und $H \withunit{\meter}$ angenommen werden.

\todo{Großer Blödsinn, Formeln nochmal abtippen.}