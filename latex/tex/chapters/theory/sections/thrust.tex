\todo{static thrust vs. jet velocity (pitch speed?)}

\section{Thrust me, I'm an engine(er)}

Thrust % Schubkraft
is the propulsive force that strives to move a mass-accelerating system forwards. 
It is a reaction force (in the sense of \textsc{Newton}'s \textit{Actio est Reactio})
that exists only in reaction to the acceleration of a mass, which happens to be a generalized fluid -- air -- in our case.
Thrust is what makes propeller engines, jet propulsion engines, ion drives and garden 
hoses
%\footnote{Garden hoses effectively are affordable jet propulsion engines.}
move forward.

In the next \namecref{subsec:momentum_theory}, 
the \textit{actuator disk} model shown in 
\cref{fig:streamtube_concept,fig:actuator_disc} \vpageref{fig:streamtube_concept,fig:actuator_disc} 
will be used for the discussion of our propeller-based system:
%
Air at atmospheric pressure $p_0 = p_\infty$ and velocity $v_0$ at location \ncircled{0} will be sucked into our propeller engine; 
this is called the \textit{inflow}. 
While it approaches the rotor area at location \ncircled{1}, air velocity increases towards $v_1 = v_i$ (the \textit{induced velocity}); due to energy conservation, pressure decreases towards $p_1$.
%This is due to the fact that the propeller effectively removes air from the inflow
%environment by "shoveling" it away, creating an underpressure, 
%which will be explained in terms of energy conservation in \cref{subsec:momentum_theory}.
Across the rotor (between \ncircled{1} and \ncircled{2}), air pressure is increased by an amount of $\Delta p$ and is promoted
to the \textit{outflow} at location \ncircled{2}; 
however, since the rotor disk is assumed to be infinitesimally thin, air velocity does not change throughout the disk, so that $v_2 = v_1 = v_i$.
While velocity keeps rising towards $v_3$ between locations \ncircled{2} and \ncircled{3},
pressure falls again until it reaches atmospheric pressure $p_3 = p_\infty$ at location \ncircled{3}.
Since energy and momentum of the rotating propeller is promoted to the air, this total increase in air velocity
$\Delta v = v_3 - v_0$ -- due to the increase in pressure $\Delta p$ -- essentially accounts for the resulting thrust.



\begin{figure}
	\begin{tikzpicture}

\newcommand{\streamtube}{
    \draw (-5,0.5) .. controls (-4,0.5) and (-5,0.5) .. (-4,0.5) .. controls (0,0.5) and (0,-0.5) .. (4,-0.5) .. controls (5,-0.5) and (4,-0.5) .. (5,-0.5);
}

% Schematische Darstellung der Strahltheorie (momentum theory)

% oberer Teil der stream tube
\begin{scope}[shift={(0,1.5)}]
	\streamtube;
\end{scope}

% unterer Teil der stream tube
\begin{scope}[shift={(0,-1.5)},yscale=-1]
	\streamtube;
\end{scope}

% Propellerebene (actuator disk)
\coordinate (propeller_upper) at (0,1.5);
\coordinate (propeller_upper_quarter) at (0,0.75);
\coordinate (propeller_center) at (0,0);
\coordinate (propeller_lower_quarter) at (0,-0.75);
\coordinate (propeller_lower) at (0,-1.5);
\draw [line width=1]  (propeller_upper) node (actuator_disk_label) [above=0.5,align=left] {actuator disk} -- 
                                    (propeller_center) -- 
                                    (propeller_lower);

\draw [->] (actuator_disk_label) -- ($(propeller_upper)+(0,0.2)$);

% propeller hub
\draw [fill=white] 
           ($(propeller_center)+(-0.2,0)$) -- 
           ($(propeller_center)+(0.1,0.15)$) -- 
           ($(propeller_center)+(0.1,-0.15)$) --
           ($(propeller_center)+(-0.2,0)$);

% upstream event horizon
\coordinate (upstream_upper) at (-5,2);
\coordinate (upstream_upper_quarter) at (-5,1);
\coordinate (upstream_center) at (-5,0);
\coordinate (upstream_lower_quarter) at (-5,-0.75);
\coordinate (upstream_lower) at (-5,-2);
\draw [dotted]  (upstream_upper) -- 
                                    (upstream_center) -- 
                                    (upstream_lower);

% downstream event horizon
\coordinate (downstream_upper) at (5,1);
\coordinate (downstream_upper_quarter) at (5,0.5);
\coordinate (downstream_center) at (5,0);
\coordinate (downstream_lower_quarter) at (5,-0.75);
\coordinate (downstream_lower) at (5,-1);
\draw [dotted]  (downstream_upper) -- 
                                    (downstream_center) -- 
                                    (downstream_lower);

% streamtube label
%\node (streamtube_label) at (upstream_upper) [above right=5.6] {streamtube};

% flow arrow
\draw [dotted] ($(upstream_center)+(0.25,0)$) -- 
                         ($(downstream_center)-(0.25,0)$);

% upstream radius
\draw [|<->|] ($(upstream_upper)-(0.25,0)$) -- 
			 ($(upstream_upper_quarter)-(0.25,0)$) node [left] {$r_u$} --
                     ($(upstream_center)-(0.25,0)$);

% downstream radius
\draw [|<->|] ($(downstream_upper)+(0.25,0)$) -- 
			 ($(downstream_upper_quarter)+(0.25,0)$) node [right] {$r_d$} --
                     ($(downstream_center)+(0.25,0)$);

% propeller radius
\draw [|<->|] ($(propeller_upper)-(0.75,0)$) -- 
			 ($(propeller_upper_quarter)-(0.75,0)$) node [left] {$r_r$} --
                     ($(propeller_center)-(0.75,0)$);

% area markers
\node [draw,circle,inner sep=0.5,outer sep=2.5] (area_0) at ($(upstream_lower)-(0,0.5)$) {0};
\node [draw,circle,inner sep=0.5,outer sep=2.5] (area_1) at ($(propeller_lower)-(0.25,1)$) {1};
\node [draw,circle,inner sep=0.5,outer sep=2.5] (area_2) at ($(propeller_lower)-(-0.25,1)$) {2};
\node [draw,circle,inner sep=0.5,outer sep=2.5] (area_3) at ($(downstream_lower)-(0,1.5)$) {3};

% inflow and outflow labels
\draw [|-|] (area_0) -- (area_1) node [below,midway] {inflow};
\draw [|-|] (area_2) -- (area_3) node [below,midway] {outflow};

% induced pressure vector
\draw [cyan,|<->|,line width=1] 
        ($(propeller_upper_quarter)-(0.25,0)$) --
        ($(propeller_upper_quarter)+(0.25,0)$)  node [right] {$\Delta p = p_2 - p_1$};

% induced velocity vector
\draw [red,->,line width=1] 
        ($(propeller_lower_quarter)-(0.25,0)$) node [left] {$v_i = v_1$} node [cyan,below left] {$p_1$} --
        ($(propeller_lower_quarter)+(0.25,0)$) node [right] {$v_2 = v_i$} node [cyan,below right] {$p_2$} ;

% upstream velocity vector
\draw [red,->,line width=1] 
        ($(upstream_lower_quarter)-(0.25,0)$) node [left] {$v_0$} node [cyan,below left] {$p_\infty = p_0$} --
        ($(upstream_lower_quarter)+(0.25,0)$);
        
% downstream velocity vector
\draw [red,->,line width=1] 
        ($(downstream_lower_quarter)-(0.25,0)$) --
        ($(downstream_lower_quarter)+(0.25,0)$) node [right] {$v_3 > v_0$} node [cyan,below right] {$p_3 = p_\infty$};

% inflow stream lines
\begin{scope}[black!50,dotted]
    \draw [->] (-4.5,1.5) .. controls (-2.5,1.5) and (-1,1) .. (-0.5,1);
    \draw [->] (-4.5,0.75) .. controls (-3,0.75) and (-2,0.5) .. (-0.5,0.5);
    \draw [->] (-4.5,-1.5) .. controls (-2.5,-1.5) and (-1.625,-1) .. (-0.875,-1);
    \draw [->] (-4.5,-0.75) .. controls (-3,-0.75) and (-2,-0.5) .. (-0.5,-0.5);
\end{scope}

\end{tikzpicture}

	\rule{35em}{0.5pt}
	\caption[Actuator disk concept of momentum theory]
			{Actuator disk concept of momentum theory: Visualization of the streamtube,
			the actuator disk and the segments of conserved energy in the ranges of inflow (\ncircled{0}-\ncircled{1}) and outflow (\ncircled{2}-\ncircled{3}) by means of air pressure $p$ and velocity $v$. \\ The distance \ncircled{1}-\ncircled{2} is
			assumed to be infinitesimally small.}
	\label{fig:streamtube_concept}
\end{figure}

\begin{figure}
	% from tdycyl at http://tex.stackexchange.com/questions/63370/drawing-3d-cylinder
\newcommand{\rotorcylinder}[5]{% origin x, origin y, origin z, radius, height
    \path (1,0,0);
    \pgfgetlastxy{\cylxx}{\cylxy}
    \path (0,1,0);
    \pgfgetlastxy{\cylyx}{\cylyy}
    \path (0,0,1);
    \pgfgetlastxy{\cylzx}{\cylzy}
    \pgfmathsetmacro{\cylt}{(\cylzy * \cylyx - \cylzx * \cylyy)/ (\cylzy * \cylxx - \cylzx * \cylxy)}
    \pgfmathsetmacro{\ang}{atan(\cylt)}
    \pgfmathsetmacro{\ct}{1/sqrt(1 + (\cylt)^2)}
    \pgfmathsetmacro{\st}{\cylt * \ct}
    \filldraw[fill=white, draw opacity=0.5, fill opacity=0] (#4*\ct+#1,#4*\st+#2,#3) -- ++(0,0,#5) arc[start angle=\ang,delta angle=-180,radius=#4] -- ++(0,0,-#5) arc[start angle=\ang+180,delta angle=-180,radius=#4];
    \filldraw[fill=white, draw opacity=0.5, fill opacity=0] (#1,#2,#3+#5) circle[radius=#4];
}

\centering
\begin{minipage}[b]{0.4\textwidth}

	\tdplotsetmaincoords{75}{145}
	\begin{tikzpicture}[tdplot_main_coords]

		\draw [black, dotted, fill=black, fill opacity=0.1] (0,0) circle (3);
		\pic{rotor};

		%\rotorcylinder{0}{0}{-2}{3}{4};
		\draw [dotted, >->] (0,0,2) -- (0,0,-2);
		
		\def\radiusA{(tanh(2)+3)}
		\def\radiusB{(tanh(2)+3)}
		\def\radiusC{(tanh(1)+3)}
		\def\radiusD{(tanh(-1)+3)}
		\def\radiusE{(tanh(-2)+3)}
		\def\radiusF{(tanh(-2)+3)}
			
		\draw [draw opacity=0.3] (0,0,2) circle (\radiusA);
		\draw [draw opacity=0.3]  (0,0,1) circle (\radiusB);
		\draw [draw opacity=0.3]  (0,0,0.35) circle (\radiusC);
		\draw [draw opacity=0.3]  (0,0,-0.35) circle (\radiusD);
		\draw [draw opacity=0.3]  (0,0,-1) circle (\radiusE);
		\draw [draw opacity=0.3]  (0,0,-2) circle (\radiusF);
		
		\foreach \t in {-180,-160,...,160}{
			\def\x{cos(\t)}
			\def\y{sin(\t)}
			
			\draw [gray,draw opacity=0.2] plot [smooth,tension=0.3] coordinates {
					({\x*\radiusA},{\y*\radiusA},2)
					({\x*\radiusB},{\y*\radiusB},1) 
					({\x*\radiusC},{\y*\radiusC},0.35)
					({\x*\radiusD},{\y*\radiusD},-0.35)
					({\x*\radiusE},{\y*\radiusE},-1)
					({\x*\radiusE},{\y*\radiusE},-2)
					};
		}
			
	\end{tikzpicture}

\end{minipage}
\hfill
\begin{minipage}[b]{0.4\textwidth}

\begin{tikzpicture}
	
	\draw [<->] (0,2) node (yaxis) [above] {$+h$} -- (0,0) -- (2,0) node (p_inf) [above right,cyan] {$p_{\infty}$} -- (4,0) node (xaxis) [above right,cyan] {$\Delta p$} node (xaxis) [below right,red] {$\Delta v$};
	\draw [->] (0,0) -- (0,-2) node (xaxis) [below] {$-h$};
	\draw [cyan,dotted] (2,2) -- (2,-2);
	
	\draw [cyan] plot [smooth,tension=1] coordinates {(2,2) (1.5,0.5) (0,0)};
	\draw [cyan] plot [smooth,tension=1] coordinates {(2,-2) (2.5,-0.5) (4,0)};
	
	\draw [red,dashed] plot [smooth,tension=1] coordinates {(3.75,-2) (3,-0.5) (1,0.5) (0.25,2)};
	
\end{tikzpicture}
\end{minipage}

	\rule{35em}{0.5pt}
	\caption[Velocity and pressure in momentum theory]
			{Velocity and pressure in momentum theory: Stream tube geometry (left),
			 change in air velocity ($\Delta v$, dashed red) and
			 air pressure relative to 
			 atmospheric pressure $p_\infty$ ($\Delta p$, solid cyan).}
	\label{fig:actuator_disc}
\end{figure}




%%%%%%%%%%%%%%%%%%%%%%%%%%%%%%%%%%%%%%%%%%%%%%%%%%%%%%%%%%%%%%%%%%%%%%%%%%%%%%%%%%%%
% MOMENTUM THEORY / STRAHLTHEORIE
%%%%%%%%%%%%%%%%%%%%%%%%%%%%%%%%%%%%%%%%%%%%%%%%%%%%%%%%%%%%%%%%%%%%%%%%%%%%%%%%%%%%

\subsection{Momentum theory and the actuator disk concept}
\label{subsec:momentum_theory}

We will assume the air to be an incompressible fluid, i.e. no density difference will be experienced across the propeller area. 
Under the principle of energy conservation and given the assumption of an incompressible mass flow, \name{Bernoulli}'s equation of pressure
%
\begin{align}
p + \rho g h + \frac{1}{2}\rho v^2 = \text{const.}
\end{align}
%
may be applied to the inflow and outflow of the accelerating system \cite{seddon2002}.
Here, $p$, given in \withunit{\newton\per\square\metre}, is the pressure of the medium, 
$\rho$, in \withunit{\kilo\gram\per\cubic\metre}, is its density,
$v$, in \withunit{\metre\per\second}, the flow velocity,
$g$, in \withunitx{\sim 9.81}{\metre\per\square\second}, the gravitational acceleration and
$h$, in \withunit{\metre}, the height above a reference point.
Since for our purposes the reference height $h$ may be assumed to be zero, the term $\rho g h$ may be ommited, such that
%
\begin{align}
p + \frac{1}{2}\rho v^2 = \text{const.} \label{eq:bernoulli}
\end{align}
%
We know that the atmospheric pressure $p_\infty$ is related to the inflow velocity $v_i$ and pressure $p_i$ by
%
\begin{align}
p_\infty = p_i + \frac{1}{2}\rho v_i^2 \label{eq:bernoulli_inflow}
\end{align}
%
From \cref{eq:bernoulli_inflow} and the pressure increase $\Delta p$ across the
rotor area, we know that 
%
\begin{align}
 \parens{p_i + \frac{1}{2}\rho v_i^2 } + \Delta p &= \parens{ p_\infty + \frac{1}{2}\rho v_\infty^2 } \label{eq:bernoulli_outflow}
\end{align}
%\todo{http://en.wikipedia.org/wiki/Blade_element_momentum_theory}
%
which, by applying \cref{eq:bernoulli_inflow} again, leads to
%
\begin{align}
 \Delta p = \frac{1}{2}\rho v_\infty^2 
 \quad \Leftrightarrow \quad 
 v_\infty = \sqrt{2\frac{\Delta p}{\rho}} \label{eq:deltap_vinfty} 
\end{align}
%
\Cref{eq:deltap_vinfty} shows that the final outstream velocity $v_\infty$ directly depends on the pressure increase $\Delta p$ induced by the rotor.

Given the circular orifice area $A = \pi r^2$ in \withunit{\square\metre} that is described by the rotating propeller blades of length $r$ in \withunit{\metre}
(the so-called \textit{rotor disk}), 
we can define the volumetric flow rate\footnote{When
thinking about the garden hose, the flow volume might be easier to understand when expressed in terms of \withunitx{1000}{\litre\per\second} rather than \withunitx{1}{\cubic\metre\per\second}, which technically is the same.} % Volumenstrom
(or flow volume, given in \withunit{\cubic\metre\per\second}) by
%
\begin{align}
\dot{q} = A v_{\infty} \label{eq:flowvolume}
\end{align}
%
and the volumetric flow rate (in \withunit{\kilo\gram\per\second}) as
%
\begin{align}
\dot{m} = \rho \dot{q} \label{eq:massflow}
\end{align}
%
Where the notation $\dot{m}$ is used to denote mass from mass over time.
With this, the thrust can be calculated as \cite{durand1935}
\todo{nah! we need a better reason. where does this equation come from? F= ...what?}
\todo{also, is $\dot{m}=\rho q$?}
%
\begin{align}
T &= \dot{m} \left( v_{\infty} - v_{self} \right) \\
  &= \rho A v_{\infty} \left( v_{\infty} - v_{self} \right) \label{eq:thrust}
\end{align}
%
where $v_{self}$ is the velocity of the propeller engine itself. 
Essentially, if the outflow velocity is larger than the velocity
of the accelerating system, then the propulsion force is positive, which in
turn will accelerate the system as a whole.

\todo{now if the system is moving faster than its jet velocity $v_{out}$, does that mean the thrust is negative and thus makes the system slower?}

If the system itself is not moving, i.e. $v_{self} = 0$, then \cref{eq:thrust} simplifies to
%
\begin{align}
T_{0} &= -\rho A \parens{v_{out}}^2 \label{eq:static_thrust_Av}
\end{align}
%
This is called the static thrust\index{thrust!static}. % Standschub

Given \cref{{eq:deltap_vinfty}} this can be rewritten to
%
\begin{align}
T_{0} 
                 &= -\rho A \parens{\sqrt{2\frac{\Delta p}{\rho}}}^2 
			  	  = -2A \Delta p
\end{align}
%
stating that if the system increases the pressure of the mass to an amount
larger than the atmospheric pressure, then this will result in a propulsing
force proportional to the area of the jet stream and the pressure difference
induced by the propeller.

As \cref{eq:static_thrust_Av} is directly dependent on
the flow volume given in \cref{eq:flowvolume}, it can be seen that there are
different ways to obtain the same amount of (static) thrust: The system may either have

\begin{itemize}
	\item a small orifice area $A$ with a high outflow velocity $\vec{v}_{out}$, or
	\item a large orifice area $A$ with a small outflow velocity $\vec{v}_{out}$.
\end{itemize}

More specifically, whenever the area $A$ doubles, the velocity $\vec{v}_{out}$ may halve. 
We will come back to this later, as this has an interesting impact on actuator design decisions, specifically in the choice of motor and propeller pairs.

\todo{come back to this later and explain the rotor and motor combinations}

\todo[size=\Large,color=red]{Now how does this relate to rotor speed?}


\begin{quote}
Ein Senkrechtstarter kann nur dann senkrecht abheben, wenn die Schubkraft größer ist als die Gewichtskraft des Flugzeugs, siehe auch Schub-Gewicht-Verhältnis. Bei einem 17 Tonnen schweren Hawker Siddeley Harrier z. B. reichen die 200 kN aus seinem Triebwerk aus, um ihn vertikal zu beschleunigen. Bei Starrflügelflugzeugen muss der Schub nur einen Bruchteil des Eigengewichts betragen, da der Flügel den anderen Teil des Eigengewichtes „trägt“.
\end{quote}
\todo{http://de.wikipedia.org/wiki/Schub}


\section{Static thrust}

\todo{two different types of thrust, static and ...}

Static thrust\index{thrust!static} % Standschub
is the thrust force applied by an unmoved (\textit{static}) system to its environment.

In case of a propeller based system, static thrust is the amount of thrust 
produced by the propeller when it is stationary in relation to the ground.

Viewed differently, it describes the force required to keep a system static if its 
thrusters operate with full power. \todo{what is it good for}

\todo{static thrust is also what makes a helicopter hover; the static thrust must at least overpower the gravitational force}

\todo{thrust-to-weight ratio}
\todo{https://quadcopterproject.wordpress.com/static-thrust-calculation/}
\todo{http://electricrcaircraftguy.blogspot.de/2013/09/propeller-static-dynamic-thrust-equation.html}

\bigbreak

Aus der Drehzahl 
$N \withunit{\per\minute}$ 
des Motors, sowie des Radius 
$r \withunit{\metre}$ 
der Luftschraube lässt sich die Winkelgeschwindigkeit 
$\omega_R \withunit{\radian\per\second}$, 
sowie die Blattspitzen- bzw. Umfangsgeschwindigkeit 
$U \withunit{\metre\per\second}$ 
ermitteln:

\begin{align}
	\omega_R &= 2\pi \cdot \frac{N}{60 \si{\per\second}} \approx 0.105 \cdot N \\
	U &= \omega_R \cdot r
\end{align}

\todo{Quellen aus \url{http://www.rc-network.de/magazin/artikel_02/art_02-0037/Standschub.pdf}}
\bigbreak

Das die Luftschraube treibende Drehmoment $M$ $\withunit{\newton\meter}$ ist mit der Antriebs-/Wellenleistung $P_W \withunit{\watt}$ des Motors über den Zusammenhang

\begin{align}
	R_W = M \cdot \omega_R
\end{align}

gekoppelt.

\bigbreak \todo{Zusammenhang...?}

Die vom Motor auf den Propeller übertragene Schubleistung $P_S$ lässt sich über die Formel

\begin{align}
	P_S = K_p \cdot \left(\frac{N}{1000}\right)^{K_t}
\end{align}

ermitteln. $K_p$ entspricht hierbei der Propellerkonstante, $K_t$ einer Leistungskonstante \todo{Quelle finden, \url{http://quadcopterproject.wordpress.com/static-thrust-calculation/} ist da sehr vage} und $N \withunit{\per\minute}$ der bekannten Drehzahl des Motors.

\bigbreak

Alternativ kann unter Vorgabe des Durchmessers $D \withunit{\inch}$ der Luftschraube und ihrer Steigung $S \withunit{\inch}$ die Leistung bei Propellerkonstanten $1.1 \leq K_p \leq 1.3$ abgeschätzt werden\todo{Quelle auf \url{http://corsair.flugmodellbau.de/files/proggies/PROPSC~1.TXT}}:

\begin{align}
	P_S = K_p \cdot \left(\frac{D}{12 \si{\inch\per\foot}}\right)^4 \cdot \frac{S}{12 \sfrac{in}{ft}} \cdot \left(\frac{N}{1000}\right)^3
\end{align}

Für eine $14\times 10$-Luftschraube mit den Werten $K_p = 1.118$ und $K_t = 3.20$ bei $N = 12000 \si{\per\minute}$ \todo{Beispiel für APC Electric $14\times 10$ auf \url{http://aircraft-world.com/prod_datasheets/hp/emeter/hp-propconstants.htm}} ergäbe sich folglich:

\begin{align}
	P_{K_p,K_t} &= 1.118 \cdot 12^{3.20} \\ &= 3175.6 \si{\watt} \notag\\
	P_{approx} &= 1.118 \cdot \left(\frac{14}{12}\right)^4 \left(\frac{10}{12}\right) \cdot \left(\frac{12000}{1000}\right)^3 \\ &= 2982.5 \si{\watt} \notag
\end{align}

Für kleinere Werte von $K_p$ weichen die Ergebnisse jedoch stark voneinander ab, so dass das Beispiel $K_p = 0.187$ und $K_t = 3.30$ bei $N = 12000 \si{\per\minute}$ \todo{APC SLOWFLY $10\times 4.7$} nicht funktioniert:

\begin{align}
	P_{K_p,K_t} &= 0.187 \cdot 12^{3.30} \\ &= 681 \si{\watt} \notag\\
	P_{approx} &= 0.187 \cdot \left(\frac{14}{12}\right)^4 \left(\frac{10}{12}\right) \cdot \left(\frac{12000}{1000}\right)^3 \\ &= 61 \si{\watt} \notag
\end{align}

\bigbreak

Gemäß \cite{standschub} kann der Schub einer Luftschraube wie folgt ermittelt werden:

\begin{align}
	K_p &= 0.0856 \cdot \frac{H}{D} - 0.0091 \\
	P_S &= K_p \cdot \rho \left(\frac{N}{60}\right)^3 \cdot D^5 \\
	F_S &= 0.67 \sqrt[3]{\frac{\rho}{2}\cdot\pi \cdot D^2 \cdot P_S^2}
\end{align}

wobei $H,D \withunit{\centi\meter}$ und $H \withunit{\meter}$ angenommen werden.

\todo{Großer Blödsinn, Formeln nochmal abtippen.}