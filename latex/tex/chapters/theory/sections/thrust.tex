\todo{static thrust vs. jet velocity (pitch speed?)}

\section{Thrust me, I'm an engine(er)}

Thrust % Schubkraft
is the propulsive force that strives to move a mass-accelerating system forwards. 
It is a reaction force (in the sense of \textsc{Newton}'s \textit{Actio est Reactio})
that exists only in reaction to the acceleration of a mass, which happens to be a generalized fluid -- air -- in our case.
Thrust is what makes propeller engines, jet propulsion engines, ion drives and garden 
hoses
%\footnote{Garden hoses effectively are affordable jet propulsion engines.}
move forward.

In the next \namecref{subsec:momentum_theory}, 
the \textit{actuator disk} model shown in 
\cref{fig:streamtube_concept,fig:actuator_disc} \vpageref{fig:streamtube_concept,fig:actuator_disc} 
will be used for the discussion of our propeller-based system:
%
Air from the \textit{free stream} at atmospheric pressure $p_0 = p_\infty$ and velocity $v_0$ (location \ncircled{0}) will be sucked into our propeller engine; 
this is called the \textit{inflow}. 
While it approaches the rotor area at location \ncircled{1}, air velocity increases towards $v_1 = v_i$ (the \textit{induced velocity}) while due to energy conservation, pressure decreases towards $p_1$.
%This is due to the fact that the propeller effectively removes air from the inflow
%environment by "shoveling" it away, creating an underpressure, 
%which will be explained in terms of energy conservation in \cref{subsec:momentum_theory}.
Across the rotor disk (between \ncircled{1} and \ncircled{2}), pressure will be increased by an amount of $\Delta p$ while air is promoted
to the \textit{outflow} at location \ncircled{2}. 
Since the rotor disk is assumed to be infinitesimally thin, however, air velocity does not change throughout the disk, so that $v_2 = v_1 = v_i$.
While velocity keeps rising towards $v_3$ between locations \ncircled{2} and \ncircled{3},
pressure falls again until it reaches atmospheric pressure $p_3 = p_\infty$ at location \ncircled{3}.
Since energy and momentum of the rotating propeller is promoted to the air, this total increase in air velocity
$\Delta v = v_3 - v_0$ -- due to the increase in pressure $\Delta p$ -- essentially accounts for the resulting thrust.



\begin{figure}
	\begin{tikzpicture}

\newcommand{\streamtube}{
    \draw (-5,0.5) .. controls (-4,0.5) and (-5,0.5) .. (-4,0.5) .. controls (0,0.5) and (0,-0.5) .. (4,-0.5) .. controls (5,-0.5) and (4,-0.5) .. (5,-0.5);
}

% Schematische Darstellung der Strahltheorie (momentum theory)

% oberer Teil der stream tube
\begin{scope}[shift={(0,1.5)}]
	\streamtube;
\end{scope}

% unterer Teil der stream tube
\begin{scope}[shift={(0,-1.5)},yscale=-1]
	\streamtube;
\end{scope}

% Propellerebene (actuator disk)
\coordinate (propeller_upper) at (0,1.5);
\coordinate (propeller_upper_quarter) at (0,0.75);
\coordinate (propeller_center) at (0,0);
\coordinate (propeller_lower_quarter) at (0,-0.75);
\coordinate (propeller_lower) at (0,-1.5);
\draw [line width=1]  (propeller_upper) node (actuator_disk_label) [above=0.5,align=left] {actuator disk} -- 
                                    (propeller_center) -- 
                                    (propeller_lower);

\draw [->] (actuator_disk_label) -- ($(propeller_upper)+(0,0.2)$);

% propeller hub
\draw [fill=white] 
           ($(propeller_center)+(-0.2,0)$) -- 
           ($(propeller_center)+(0.1,0.15)$) -- 
           ($(propeller_center)+(0.1,-0.15)$) --
           ($(propeller_center)+(-0.2,0)$);

% upstream event horizon
\coordinate (upstream_upper) at (-5,2);
\coordinate (upstream_upper_quarter) at (-5,1);
\coordinate (upstream_center) at (-5,0);
\coordinate (upstream_lower_quarter) at (-5,-0.75);
\coordinate (upstream_lower) at (-5,-2);
\draw [dotted]  (upstream_upper) -- 
                                    (upstream_center) -- 
                                    (upstream_lower);

% downstream event horizon
\coordinate (downstream_upper) at (5,1);
\coordinate (downstream_upper_quarter) at (5,0.5);
\coordinate (downstream_center) at (5,0);
\coordinate (downstream_lower_quarter) at (5,-0.75);
\coordinate (downstream_lower) at (5,-1);
\draw [dotted]  (downstream_upper) -- 
                                    (downstream_center) -- 
                                    (downstream_lower);

% streamtube label
%\node (streamtube_label) at (upstream_upper) [above right=5.6] {streamtube};

% flow arrow
\draw [dotted] ($(upstream_center)+(0.25,0)$) -- 
                         ($(downstream_center)-(0.25,0)$);

% upstream radius
\draw [|<->|] ($(upstream_upper)-(0.25,0)$) -- 
			 ($(upstream_upper_quarter)-(0.25,0)$) node [left] {$r_u$} --
                     ($(upstream_center)-(0.25,0)$);

% downstream radius
\draw [|<->|] ($(downstream_upper)+(0.25,0)$) -- 
			 ($(downstream_upper_quarter)+(0.25,0)$) node [right] {$r_d$} --
                     ($(downstream_center)+(0.25,0)$);

% propeller radius
\draw [|<->|] ($(propeller_upper)-(0.75,0)$) -- 
			 ($(propeller_upper_quarter)-(0.75,0)$) node [left] {$r_r$} --
                     ($(propeller_center)-(0.75,0)$);

% area markers
\node [draw,circle,inner sep=0.5,outer sep=2.5] (area_0) at ($(upstream_lower)-(0,0.5)$) {0};
\node [draw,circle,inner sep=0.5,outer sep=2.5] (area_1) at ($(propeller_lower)-(0.25,1)$) {1};
\node [draw,circle,inner sep=0.5,outer sep=2.5] (area_2) at ($(propeller_lower)-(-0.25,1)$) {2};
\node [draw,circle,inner sep=0.5,outer sep=2.5] (area_3) at ($(downstream_lower)-(0,1.5)$) {3};

% inflow and outflow labels
\draw [|-|] (area_0) -- (area_1) node [below,midway] {inflow};
\draw [|-|] (area_2) -- (area_3) node [below,midway] {outflow};

% induced pressure vector
\draw [cyan,|<->|,line width=1] 
        ($(propeller_upper_quarter)-(0.25,0)$) --
        ($(propeller_upper_quarter)+(0.25,0)$)  node [right] {$\Delta p = p_2 - p_1$};

% induced velocity vector
\draw [red,->,line width=1] 
        ($(propeller_lower_quarter)-(0.25,0)$) node [left] {$v_i = v_1$} node [cyan,below left] {$p_1$} --
        ($(propeller_lower_quarter)+(0.25,0)$) node [right] {$v_2 = v_i$} node [cyan,below right] {$p_2$} ;

% upstream velocity vector
\draw [red,->,line width=1] 
        ($(upstream_lower_quarter)-(0.25,0)$) node [left] {$v_0$} node [cyan,below left] {$p_\infty = p_0$} --
        ($(upstream_lower_quarter)+(0.25,0)$);
        
% downstream velocity vector
\draw [red,->,line width=1] 
        ($(downstream_lower_quarter)-(0.25,0)$) --
        ($(downstream_lower_quarter)+(0.25,0)$) node [right] {$v_3 > v_0$} node [cyan,below right] {$p_3 = p_\infty$};

% inflow stream lines
\begin{scope}[black!50,dotted]
    \draw [->] (-4.5,1.5) .. controls (-2.5,1.5) and (-1,1) .. (-0.5,1);
    \draw [->] (-4.5,0.75) .. controls (-3,0.75) and (-2,0.5) .. (-0.5,0.5);
    \draw [->] (-4.5,-1.5) .. controls (-2.5,-1.5) and (-1.625,-1) .. (-0.875,-1);
    \draw [->] (-4.5,-0.75) .. controls (-3,-0.75) and (-2,-0.5) .. (-0.5,-0.5);
\end{scope}

\end{tikzpicture}

			\rule{35em}{0.5pt}
	\caption[Actuator disk concept of momentum theory]
			{Actuator disk concept of momentum theory: Visualization of the streamtube control volume,
			the actuator disk and the segments of conserved energy in the ranges of inflow (\ncircled{0}-\ncircled{1}) and outflow (\ncircled{2}-\ncircled{3}) by means of air pressure $p$ and velocity $v$. \\ The distance \ncircled{1}-\ncircled{2} is
			assumed to be infinitesimally small.}
	\label{fig:streamtube_concept}
\end{figure}

\begin{figure}
	% from tdycyl at http://tex.stackexchange.com/questions/63370/drawing-3d-cylinder
\newcommand{\rotorcylinder}[5]{% origin x, origin y, origin z, radius, height
    \path (1,0,0);
    \pgfgetlastxy{\cylxx}{\cylxy}
    \path (0,1,0);
    \pgfgetlastxy{\cylyx}{\cylyy}
    \path (0,0,1);
    \pgfgetlastxy{\cylzx}{\cylzy}
    \pgfmathsetmacro{\cylt}{(\cylzy * \cylyx - \cylzx * \cylyy)/ (\cylzy * \cylxx - \cylzx * \cylxy)}
    \pgfmathsetmacro{\ang}{atan(\cylt)}
    \pgfmathsetmacro{\ct}{1/sqrt(1 + (\cylt)^2)}
    \pgfmathsetmacro{\st}{\cylt * \ct}
    \filldraw[fill=white, draw opacity=0.5, fill opacity=0] (#4*\ct+#1,#4*\st+#2,#3) -- ++(0,0,#5) arc[start angle=\ang,delta angle=-180,radius=#4] -- ++(0,0,-#5) arc[start angle=\ang+180,delta angle=-180,radius=#4];
    \filldraw[fill=white, draw opacity=0.5, fill opacity=0] (#1,#2,#3+#5) circle[radius=#4];
}

\centering
\begin{minipage}[b]{0.4\textwidth}

	\tdplotsetmaincoords{75}{145}
	\begin{tikzpicture}[tdplot_main_coords]

		\draw [black, dotted, fill=black, fill opacity=0.1] (0,0) circle (3);
		\pic{rotor};

		%\rotorcylinder{0}{0}{-2}{3}{4};
		\draw [dotted, >->] (0,0,2) -- (0,0,-2);
		
		\def\radiusA{(tanh(2)+3)}
		\def\radiusB{(tanh(2)+3)}
		\def\radiusC{(tanh(1)+3)}
		\def\radiusD{(tanh(-1)+3)}
		\def\radiusE{(tanh(-2)+3)}
		\def\radiusF{(tanh(-2)+3)}
			
		\draw [draw opacity=0.3] (0,0,2) circle (\radiusA);
		\draw [draw opacity=0.3]  (0,0,1) circle (\radiusB);
		\draw [draw opacity=0.3]  (0,0,0.35) circle (\radiusC);
		\draw [draw opacity=0.3]  (0,0,-0.35) circle (\radiusD);
		\draw [draw opacity=0.3]  (0,0,-1) circle (\radiusE);
		\draw [draw opacity=0.3]  (0,0,-2) circle (\radiusF);
		
		\foreach \t in {-180,-160,...,160}{
			\def\x{cos(\t)}
			\def\y{sin(\t)}
			
			\draw [gray,draw opacity=0.2] plot [smooth,tension=0.3] coordinates {
					({\x*\radiusA},{\y*\radiusA},2)
					({\x*\radiusB},{\y*\radiusB},1) 
					({\x*\radiusC},{\y*\radiusC},0.35)
					({\x*\radiusD},{\y*\radiusD},-0.35)
					({\x*\radiusE},{\y*\radiusE},-1)
					({\x*\radiusE},{\y*\radiusE},-2)
					};
		}
			
	\end{tikzpicture}

\end{minipage}
\hfill
\begin{minipage}[b]{0.4\textwidth}

\begin{tikzpicture}
	
	\draw [<->] (0,2) node (yaxis) [above] {$+h$} -- (0,0) -- (2,0) node (p_inf) [above right,cyan] {$p_{\infty}$} -- (4,0) node (xaxis) [above right,cyan] {$\Delta p$} node (xaxis) [below right,red] {$\Delta v$};
	\draw [->] (0,0) -- (0,-2) node (xaxis) [below] {$-h$};
	\draw [cyan,dotted] (2,2) -- (2,-2);
	
	\draw [cyan] plot [smooth,tension=1] coordinates {(2,2) (1.5,0.5) (0,0)};
	\draw [cyan] plot [smooth,tension=1] coordinates {(2,-2) (2.5,-0.5) (4,0)};
	
	\draw [red,dashed] plot [smooth,tension=1] coordinates {(3.75,-2) (3,-0.5) (1,0.5) (0.25,2)};
	
\end{tikzpicture}
\end{minipage}

	\rule{35em}{0.5pt}
	\caption[Velocity and pressure in momentum theory]
			{Velocity and pressure in momentum theory: Stream tube geometry (left),
			 change in air velocity ($\Delta v$, dashed red) and
			 air pressure relative to 
			 atmospheric pressure $p_\infty$ ($\Delta p$, solid cyan).}
	\label{fig:actuator_disc}
\end{figure}

Note that \cref{fig:streamtube_concept} relates a static actuator disk (of $\vec{v}_{disk} = 0$) to a nonzero free stream velocity $\vec{v}_0$. 
This, by relativity of motion, is conceptually the same as relating a nonstatic actuator disk to a static free stream of velocity $\vec{v}_0 = 0$. 
Because of this, we will use $\vec{v}_0$ in the following to denote motion of the actuator disk itself within a static environment.


%%%%%%%%%%%%%%%%%%%%%%%%%%%%%%%%%%%%%%%%%%%%%%%%%%%%%%%%%%%%%%%%%%%%%%%%%%%%%%%%%%%%
% MOMENTUM THEORY / STRAHLTHEORIE
%%%%%%%%%%%%%%%%%%%%%%%%%%%%%%%%%%%%%%%%%%%%%%%%%%%%%%%%%%%%%%%%%%%%%%%%%%%%%%%%%%%%

\subsection{Momentum theory and static thrust}
\label{subsec:momentum_theory}

We will assume the air to be an incompressible fluid, i.e. no density difference will be experienced across the propeller area. 
Under the principle of energy conservation and given the assumption of an incompressible mass flow, \name{Bernoulli}'s equation of pressure
%
\begin{align}
p + \rho g h + \frac{1}{2}\rho v^2 = \text{const.}
\end{align}
%
may be applied to the inflow (\ncircled{0}-\ncircled{1}) and outflow (\ncircled{2}-\ncircled{3}) sections of the accelerating system \cite{seddon2002}.
Here, $p$, given in \withunit{\newton\per\square\metre}, is the air pressure, 
$\rho$, in \withunit{\kilo\gram\per\cubic\metre}, its density,
$v$, in \withunit{\metre\per\second}, the flow velocity,
$g$, in \withunitx{\sim 9.81}{\metre\per\square\second}, the gravitational acceleration and
$h$, in \withunit{\metre}, the height above a reference point.
%
\todo{why may be ignore the $\rho g h$ term? negligible compared to rotor pressure?}
%
Since for our purposes the term $\rho g h$ can be assumed to be negligible, it may be ommited, such that
%
\begin{align}
p + \frac{1}{2} \rho v^2 = \text{const.} \label{eq:bernoulli}
\end{align}

If we apply \cref{eq:bernoulli} to the inflow section, due to energy conservation we obtain
%
\begin{align}
p_\infty + \frac{1}{2} \rho v^2_0 &= p_1 + \frac{1}{2} \rho v^2_i \label{eq:bernoulli_inflow}
\end{align}
%
where we use the equivalence $p_0 = p_\infty$, similarly for the outflow section, we get
%
\begin{align}
p_2 + \frac{1}{2} \rho v^2_i &= p_\infty + \frac{1}{2} \rho v^2_3 \label{eq:bernoulli_outflow}
\end{align}
%
again using $p_3 = p_\infty$.

If we express $p_2$ in terms of $p_1$ and a pressure difference $\Delta p$, such that
\begin{align}
p_2 &= \Delta p + p_1
\end{align}
%
we find that
%
\begin{align}
\Delta p + \parens{ p_1 + \frac{1}{2} \rho v^2_i } &= \parens{  p_2 + \frac{1}{2} \rho v^2_i }
\end{align}
%
By identity using \cref{eq:bernoulli_inflow,eq:bernoulli_outflow} we obtain
%
\begin{align}
\Delta p + \parens{ p_\infty + \frac{1}{2} \rho v^2_0 } &= \parens{ p_\infty + \frac{1}{2} \rho v^2_3 } \label{eq:bernoulli_induced_deltap}
\end{align}
%
and thus%
\footnote{Using \cref{eq:bernoulli_inflow,eq:bernoulli_outflow}, calculating
$\parens{ p_2 - p_1 } = \parens{p_\infty + \frac{1}{2} \rho v^2_3 - \frac{1}{2} \rho v^2_i} - \parens{ p_\infty + \frac{1}{2} \rho v^2_0 - \frac{1}{2} \rho v^2_i}$ yields the same result.
}
%
\begin{align}
\Delta p &= \frac{1}{2} \rho \parens{ v^2_3 - v^2_0 }
\end{align}

Using actuator disk concept, we assume that the volume swept out by the rotating propeller blades describes an infinitesimally thin disk (the so-called \textit{rotor disk}) with orifice area $A = 2\pi r$%
% and volume $V_d = A s \sfrac{\diff}{\diff s}$
, where $r$ (in \withunit{\metre}) is the length of the propeller blades (i.e. the radius of the propeller).
Because of the disk being infinitesimally thin, the air velocity across the disk can be assumed to be continuous, i.e. the change in air velocity is $\Delta v = 0$. 

Given the orifice area $A$, we can then define the volumetric flow rate % Volumenstrom
(or flow volume, given in \withunit{\cubic\metre\per\second})%
\footnote{When
thinking about the garden hose, the flow volume might be easier to understand when expressed in terms of \withunitx{1000}{\litre\per\second} rather than \withunitx{1}{\cubic\metre\per\second}, which technically is the same.}
as the flow of air mass through the disk by
%
\begin{align}
Q = \dot{V} = A v_i \label{eq:flowvolume}
\end{align}
%
where $\dot{V} = \sfrac{\diff V}{\diff t}$, as well as the mass flow rate (in \withunit{\kilo\gram\per\second}) as
%
\begin{align}
\dot{m} = \rho Q = \rho A v_i \label{eq:massflow}
\end{align}
%
where the notation $\dot{m}$ is used to denote mass over time, i.e. $\sfrac{\diff m}{\diff t}$. % Here, $\rho \cdot v$ is the momentum density.

Since mass enters and leaves the control volume only through upstream and downstream orifice areas $A_u = 2 \pi r_u$ and $A_d = 2 \pi r_d$ (compare \cref{fig:streamtube_concept}), continuity equation gives that 
%
\begin{align}
\dot{m} = \rho A_u v_0 = \rho A v_i = \rho A_d v_3 \label{eq:massflow_continuity}
\end{align}

Further, momentum and energy induced by the rotor must be conserved. 
Since $\dot{m}$ does not change, the eventual outflow pressure $p_\infty$ and velocity $v_3$ are a function of the rotor energy and can thus be used
to describe the force acting on the outflow through the rotor.

\begin{figure}
	\centering
	\begin{tikzpicture}[scale=1]

	\coordinate (disk_upper) at (0,1);
	\coordinate (disk_lower) at (0,-1);

	% the disk
	\draw [line width=2] 
		(disk_lower) -- 
		(disk_upper) node (center) [midway, inner sep = 0] {};
		
	% force arrow
	\draw [->,line width=1,magenta] 
		(center) -- 
		++ (1,0) node [right] {$\vec{F} = \dotvec{m} \vec{v}$};

	% disk radius
	\draw [|<->|]
		($(disk_upper)+(0.25,0)$) --
		($(center)+(0.25,0)$) node [midway,right] {$r$};

	% fluid velocity vectors on the disk
	\def\length{2}
	\foreach {\x} in {0.2,0.4,...,1} {
		\draw [->,dotted]
			(-\length,\x) --
			(-0.125,\x);
		\draw [->,dotted]
			(-\length,-\x) --
			(-0.125,-\x);
	}
	\draw [->,dotted,line width=0.6]
			(-\length,0) --
			(-0.125,0) node [preaction={fill=white},inner sep=2pt,midway,red] {$\vec{v}$};
		
	% fluid velocity vectors above and below the disk
	\foreach {\x} in {1.2,1.4} {
		\draw [->,dotted]
			(-\length,\x) --
			(\length,\x);
		\draw [->,dotted]
			(-\length,-\x) --
			(\length,-\x);
	}
		
	% m-dot brace
	\draw [magenta,decorate, decoration={brace, amplitude=5pt}]
		($(disk_lower)-(\length,0)$) --
		($(disk_upper)-(\length,0)$)
		node [midway, left=5pt, align=left] {$\rho (2 \pi r) \vec{v} = \dotvec{m}$};

% m * v ==> m * Delta v

\end{tikzpicture}

	\rule{35em}{0.5pt}
	\caption[Mass flow rate and force]
			{Relation of the mass flow rate $\dotvec{m}$ acting on a disk of area $A = 2\pi r$ to the resulting force $\vec{F}$.}
	\label{fig:mass_flow_force}
\end{figure}

In order to relate the mass flow rate $\dot{m}$ towards the rotor with the force acting on the rotor (or vice versa), \name{Newton}'s second law of motion
%
\begin{align}
F &= \frac{\diff}{\diff t} \parens{ m v } \label{eq:newtons_2nd}
\end{align}
%
may be used \cite{book:siekmann2008}.
If the velocity is assumed to be static (i.e. constant), such that $\sfrac{\diff v}{\diff t} = 0$. 
As a result, the mass must be varying over time, which is the very definition of the mass flow rate $\dot{m} = \sfrac{\diff m}{\diff t}$ (compare \cref{eq:massflow}). 
This then leads to
%
\begin{align}
F &= \frac{\diff m}{\diff t} v = \dot{m} v \quad \mathrm{if} \quad v = \mathrm{const.} \label{eq:newtons_2nd_mdot}
\end{align}
%
which is displayed graphically in \cref{fig:mass_flow_force}.

Given the relative speed of air in the outflow%
\footnote{Remember that $v_0$ can be understood as the actuator disk velocity as well as the free stream velocity, depending on the point of view.} 
%
\begin{align}
\vec{v}_{rel} &= \vec{v}_3 - \vec{v}_0
\end{align}
%
as well as \cref{eq:massflow,eq:newtons_2nd_mdot}, the force applied to the air in the outflow can be calculated \cite{durand1935} as
%
\begin{align}
\vec{F} &= \dotvec{m} \vec{v}_{rel} \notag \\
		&= \dotvec{m} \pointwise \parens{ \vec{v}_3 - \vec{v}_0 } \\
		&= \rho A \vec{v}_i \pointwise \parens{ \vec{v}_3 - \vec{v}_0 } \label{eq:thrust_mdot}
\end{align}
%
\todo{that is probably bullshite; test if $\dot{m} = \rho A (\vec{v}_3 - \vec{v}_0)$ fits in better}
Here $\pointwise$ denotes the entrywise (\textit{Hadamard} or \textit{Schur}) product of two vectors, as the resulting force will still point in the same direction as all of the velocities.

\Cref{eq:thrust_mdot} essentially states that if the outflow velocity is larger than the velocity
of the accelerating system itself, the force applied on the air is positive.% 
\footnote{It should be clear that an outflow velocity of $\vec{v}_3 = \vec{v}_0$ does not do anything useful -- no energy would have been transmitted to the outflow at all -- 
which is why the resulting force would be zero.}

\todo{now if the system is moving faster than its jet velocity $v_{out}$, does that mean the thrust is negative and thus makes the system slower?}

Since we have the thrust $F_T$ defined to be the reaction force to the action of moving mass, it consequently is given using \cref{eq:thrust_mdot} as
%
\begin{align}
\vec{F}_T &= -\vec{F} \label{eq:thrust}
\end{align}
%
and points towards the inflow.

If the system itself is not moving, i.e. $v_{self} = 0$, then \cref{eq:thrust_mdot} simplifies to
%
\begin{align}
F_{0} &= \rho A v_i v_\infty \label{eq:static_thrust_Av}
\end{align}
%
This is called the static thrust\index{thrust!static}. % Standschub

\todo{also, $v_\infty = 2v_i$ according to \url{http://aerostudents.com/files/aircraftPerformance/helicopters.pdf}}

Given \cref{{eq:deltap_vinfty}} this can be rewritten to
%
\begin{align}
F_{0} 
                 &= -\rho A \parens{\sqrt{2\frac{\Delta p}{\rho}}}^2 
			  	  = -2A \Delta p
\end{align}
\todo{this doesn't follow from above anymore - patch in the $v_\infty = 2v_i$}
%
stating that if the system increases the pressure of the mass to an amount
larger than the atmospheric pressure, then this will result in a propulsing
force proportional to the area of the jet stream and the pressure difference
induced by the propeller.

As \cref{eq:static_thrust_Av} is directly dependent on
the flow volume given in \cref{eq:flowvolume}, it can be seen that there are
different ways to obtain the same amount of (static) thrust: The system may either have

\begin{itemize}
	\item a small orifice area $A$ with a high outflow velocity $\vec{v}_{out}$, or
	\item a large orifice area $A$ with a small outflow velocity $\vec{v}_{out}$.
\end{itemize}

More specifically, whenever the area $A$ doubles, the velocity $\vec{v}_{out}$ may halve. 
We will come back to this later, as this has an interesting impact on actuator design decisions, specifically in the choice of motor and propeller pairs.

\todo{come back to this later and explain the rotor and motor combinations}

\todo[size=\Large,color=red]{Now how does this relate to rotor speed?}


\begin{quote}
Ein Senkrechtstarter kann nur dann senkrecht abheben, wenn die Schubkraft größer ist als die Gewichtskraft des Flugzeugs, siehe auch Schub-Gewicht-Verhältnis. Bei einem 17 Tonnen schweren Hawker Siddeley Harrier z. B. reichen die 200 kN aus seinem Triebwerk aus, um ihn vertikal zu beschleunigen. Bei Starrflügelflugzeugen muss der Schub nur einen Bruchteil des Eigengewichts betragen, da der Flügel den anderen Teil des Eigengewichtes „trägt“.
\end{quote}
\todo{http://de.wikipedia.org/wiki/Schub}


\section{Static thrust}

\todo{two different types of thrust, static and ...}

Static thrust\index{thrust!static} % Standschub
is the thrust force applied by an unmoved (\textit{static}) system to its environment.

In case of a propeller based system, static thrust is the amount of thrust 
produced by the propeller when it is stationary in relation to the ground.

Viewed differently, it describes the force required to keep a system static if its 
thrusters operate with full power. \todo{what is it good for}

\todo{static thrust is also what makes a helicopter hover; the static thrust must at least overpower the gravitational force}

\todo{thrust-to-weight ratio}
\todo{https://quadcopterproject.wordpress.com/static-thrust-calculation/}
\todo{http://electricrcaircraftguy.blogspot.de/2013/09/propeller-static-dynamic-thrust-equation.html}

\bigbreak

Aus der Drehzahl 
$N \withunit{\per\minute}$ 
des Motors, sowie des Radius 
$r \withunit{\metre}$ 
der Luftschraube lässt sich die Winkelgeschwindigkeit 
$\omega_R \withunit{\radian\per\second}$, 
sowie die Blattspitzen- bzw. Umfangsgeschwindigkeit 
$U \withunit{\metre\per\second}$ 
ermitteln:

\begin{align}
	\omega_R &= 2\pi \cdot \frac{N}{60 \si{\per\second}} \approx 0.105 \cdot N \\
	U &= \omega_R \cdot r
\end{align}

\todo{Quellen aus \url{http://www.rc-network.de/magazin/artikel_02/art_02-0037/Standschub.pdf}}
\bigbreak

Das die Luftschraube treibende Drehmoment $M$ $\withunit{\newton\meter}$ ist mit der Antriebs-/Wellenleistung $P_W \withunit{\watt}$ des Motors über den Zusammenhang

\begin{align}
	R_W = M \cdot \omega_R
\end{align}

gekoppelt.

\bigbreak \todo{Zusammenhang...?}

Die vom Motor auf den Propeller übertragene Schubleistung $P_S$ lässt sich über die Formel

\begin{align}
	P_S = K_p \cdot \left(\frac{N}{1000}\right)^{K_t}
\end{align}

ermitteln. $K_p$ entspricht hierbei der Propellerkonstante, $K_t$ einer Leistungskonstante \todo{Quelle finden, \url{http://quadcopterproject.wordpress.com/static-thrust-calculation/} ist da sehr vage} und $N \withunit{\per\minute}$ der bekannten Drehzahl des Motors.

\bigbreak

Alternativ kann unter Vorgabe des Durchmessers $D \withunit{\inch}$ der Luftschraube und ihrer Steigung $S \withunit{\inch}$ die Leistung bei Propellerkonstanten $1.1 \leq K_p \leq 1.3$ abgeschätzt werden\todo{Quelle auf \url{http://corsair.flugmodellbau.de/files/proggies/PROPSC~1.TXT}}:

\begin{align}
	P_S = K_p \cdot \left(\frac{D}{12 \si{\inch\per\foot}}\right)^4 \cdot \frac{S}{12 \sfrac{in}{ft}} \cdot \left(\frac{N}{1000}\right)^3
\end{align}

Für eine $14\times 10$-Luftschraube mit den Werten $K_p = 1.118$ und $K_t = 3.20$ bei $N = 12000 \si{\per\minute}$ \todo{Beispiel für APC Electric $14\times 10$ auf \url{http://aircraft-world.com/prod_datasheets/hp/emeter/hp-propconstants.htm}} ergäbe sich folglich:

\begin{align}
	P_{K_p,K_t} &= 1.118 \cdot 12^{3.20} \\ &= 3175.6 \si{\watt} \notag\\
	P_{approx} &= 1.118 \cdot \left(\frac{14}{12}\right)^4 \left(\frac{10}{12}\right) \cdot \left(\frac{12000}{1000}\right)^3 \\ &= 2982.5 \si{\watt} \notag
\end{align}

Für kleinere Werte von $K_p$ weichen die Ergebnisse jedoch stark voneinander ab, so dass das Beispiel $K_p = 0.187$ und $K_t = 3.30$ bei $N = 12000 \si{\per\minute}$ \todo{APC SLOWFLY $10\times 4.7$} nicht funktioniert:

\begin{align}
	P_{K_p,K_t} &= 0.187 \cdot 12^{3.30} \\ &= 681 \si{\watt} \notag\\
	P_{approx} &= 0.187 \cdot \left(\frac{14}{12}\right)^4 \left(\frac{10}{12}\right) \cdot \left(\frac{12000}{1000}\right)^3 \\ &= 61 \si{\watt} \notag
\end{align}

\bigbreak

Gemäß \cite{standschub} kann der Schub einer Luftschraube wie folgt ermittelt werden:

\begin{align}
	K_p &= 0.0856 \cdot \frac{H}{D} - 0.0091 \\
	P_S &= K_p \cdot \rho \left(\frac{N}{60}\right)^3 \cdot D^5 \\
	F_S &= 0.67 \sqrt[3]{\frac{\rho}{2}\cdot\pi \cdot D^2 \cdot P_S^2}
\end{align}

wobei $H,D \withunit{\centi\meter}$ und $H \withunit{\meter}$ angenommen werden.

\todo{Großer Blödsinn, Formeln nochmal abtippen.}