\section{Der Schub and all his friends}

\todo{two different types of thrust, static and ...}

\emph{Static} thrust % Standschub
is the thrust applied by a system given the precondition that no movement is happening. \todo{what is it good for}

\todo{would it be correct to assume that static thrust is what makes a helicopter float?}

\bigbreak

Aus der Drehzahl 
$N \left[\si{U\per\minute}\right]$ 
des Motors, sowie des Radius 
$r \left[\si{\metre}\right]$ 
der Luftschraube lässt sich die Winkelgeschwindigkeit 
$\omega_R \left[\si{\radian\per\second}\right]$, 
sowie die Blattspitzen- bzw. Umfangsgeschwindigkeit 
$U \left[\si{\metre\per\second}\right]$ 
ermitteln:

\begin{align}
	\omega_R &= 2\pi \cdot \frac{N}{60 \si{U\per\second}} \approx 0.105 \cdot N \\
	U &= \omega_R \cdot r
\end{align}

\todo{Quellen aus \url{http://www.rc-network.de/magazin/artikel_02/art_02-0037/Standschub.pdf}}
\bigbreak

Das die Luftschraube treibende Drehmoment $M$ $\left[\si{\newton\meter}\right]$ ist mit der Antriebs-/Wellenleistung $P_W \left[\si{\watt}\right]$ des Motors über den Zusammenhang

\begin{align}
	R_W = M \cdot \omega_R
\end{align}

gekoppelt.

\bigbreak \todo{Zusammenhang...?}

Die vom Motor auf den Propeller übertragene Schubleistung $P_S$ lässt sich über die Formel

\begin{align}
	P_S = K_p \cdot \left(\frac{N}{1000}\right)^{K_t}
\end{align}

ermitteln. $K_p$ entspricht hierbei der Propellerkonstante, $K_t$ einer Leistungskonstante \todo{Quelle finden, \url{http://quadcopterproject.wordpress.com/static-thrust-calculation/} ist da sehr vage} und $N$ $\left[\si{U\per\minute}\right]$ der bekannten Drehzahl des Motors.

\bigbreak

Alternativ kann unter Vorgabe des Durchmessers $D \left[\text{inch}\right]$ der Luftschraube und ihrer Steigung $S \left[\text{inch}\right]$ die Leistung bei Propellerkonstanten $1.1 \leq K_p \leq 1.3$ abgeschätzt werden\todo{Quelle auf \url{http://corsair.flugmodellbau.de/files/proggies/PROPSC~1.TXT}}:

\begin{align}
	P_S = K_p \cdot \left(\frac{D}{12 \sfrac{in}{ft}}\right)^4 \cdot \frac{S}{12 \sfrac{in}{ft}} \cdot \left(\frac{N}{1000}\right)^3
\end{align}

Für eine $14\times 10$-Luftschraube mit den Werten $K_p = 1.118$ und $K_t = 3.20$ bei $N = 12000 \si{U\per\minute}$ \todo{Beispiel für APC Electric $14\times 10$ auf \url{http://aircraft-world.com/prod_datasheets/hp/emeter/hp-propconstants.htm}} ergäbe sich folglich:

\begin{align}
	P_{K_p,K_t} &= 1.118 \cdot 12^{3.20} \\ &= 3175.6 \si{\watt} \notag\\
	P_{approx} &= 1.118 \cdot \left(\frac{14}{12}\right)^4 \left(\frac{10}{12}\right) \cdot \left(\frac{12000}{1000}\right)^3 \\ &= 2982.5 \si{\watt} \notag
\end{align}

Für kleinere Werte von $K_p$ weichen die Ergebnisse jedoch stark voneinander ab, so dass das Beispiel $K_p = 0.187$ und $K_t = 3.30$ bei $N = 12000 \si{U\per\minute}$ \todo{APC SLOWFLY $10\times 4.7$} nicht funktioniert:

\begin{align}
	P_{K_p,K_t} &= 0.187 \cdot 12^{3.30} \\ &= 681 \si{\watt} \notag\\
	P_{approx} &= 0.187 \cdot \left(\frac{14}{12}\right)^4 \left(\frac{10}{12}\right) \cdot \left(\frac{12000}{1000}\right)^3 \\ &= 61 \si{\watt} \notag
\end{align}

\bigbreak

Gemäß \cite{standschub} kann der Schub einer Luftschraube wie folgt ermittelt werden:

\begin{align}
	K_p &= 0.0856 \cdot \frac{H}{D} - 0.0091 \\
	P_S &= K_p \cdot \rho \left(\frac{N}{60}\right)^3 \cdot D^5 \\
	F_S &= 0.67 \sqrt[3]{\frac{\rho}{2}\cdot\pi \cdot D^2 \cdot P_S^2}
\end{align}

wobei $H,D \left[\si{\centi\meter}\right]$ und $H \left[\si{\meter}\right]$ angenommen werden.

\todo{Großer Blödsinn, Formeln nochmal aptippen.}