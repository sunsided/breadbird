% Grafiken
\usepackage{graphicx}
\usepackage{float} % lädt das Paket zur Verwendung von zusätzlichen Positionsbefehlen
\usepackage{tikz} % http://cremeronline.com/LaTeX/minimaltikz.pdf
\usepackage{tikz-3dplot} % ftp://ftp.rrzn.uni-hannover.de/pub/mirror/tex-archive/graphics/pgf/contrib/tikz-3dplot/tikz-3dplot_documentation.pdf

\usepackage{calc} % for calculating coordinates within TikZ

\usepackage{esk} % http://www.bitlib.net/mirror/ctan/macros/latex/contrib/esk/esk.pdf
\usetikzlibrary{decorations.pathreplacing}
\usetikzlibrary{decorations.markings}
\usetikzlibrary{calc}
\usetikzlibrary{shapes,arrows,3D}
\usetikzlibrary{automata}

%\usetikzlibrary{external}
%\tikzexternalize[prefix=.tikz-external/]

\usepackage{rotating}

\usepackage{pgfplots}
\pgfplotsset{compat=newest}

\tikzstyle{block} = [draw, fill=blue!20, rectangle, minimum height=2em]%, minimum width=4em]
\tikzstyle{sum} = [draw, fill=blue!20, circle]%, node distance=1cm]
\tikzstyle{input} = [coordinate]
\tikzstyle{output} = [coordinate]
\tikzstyle{pinstyle} = [pin edge={to-,thin,black}]

% pstricks - Grafiken
%\usepackage{pstricks}
%\usepackage{pst-plot}
%\usepackage{pstricks-add}

\graphicspath{{pictures/}}

%----------------------------------------------------------------------------------------
%	TikZ BASED TEXT HACKS
%----------------------------------------------------------------------------------------

% http://tex.stackexchange.com/questions/7032/good-way-to-make-textcircled-numbers
\newcommand*\circled[1]{\tikz[baseline=(char.base)]{
    \node[shape=circle,draw,inner sep=2pt] (char) {#1};}}

% ncircled is narrow circle
\newcommand*\ncircled[1]{\tikz[baseline=(char.base)]{
    \node[shape=circle,draw,inner sep=1pt] (char) {#1};}}

%----------------------------------------------------------------------------------------
%	THE ROTOR 
%----------------------------------------------------------------------------------------

\tikzset{
    rotor/.pic={
    	\def\rotorcolor{black}

    	\def\offset{0.125} 
		\def\radius{0.16} 
    
        \def\width{0.5}
		\def\length{3}
		\def\belly{2} 
		
		\def\bladeopacity{0.3}
		\def\diskopacity{0.8}
		
		%\draw [gray!50]  (\offset,\offset) -- (\offset+\belly,\width) -- (\length,0) -- (\offset+\belly,-\width) -- (\offset,-\offset); 
		\draw [\rotorcolor, fill=\rotorcolor, fill opacity=\bladeopacity] plot [smooth, tension=0.8] coordinates { (\offset,\offset) (\offset+\belly,\width) (\length,0) (\offset+\belly,-\width) (\offset,-\offset)};
				
		%\draw [gray!50]  (-\offset,\offset) -- (-\offset-\belly,\width) -- (-\length,0) -- (-\offset-\belly,-\width) -- (-\offset,-\offset); 
		\draw [\rotorcolor, fill=\rotorcolor, fill opacity=\bladeopacity] plot [smooth, tension=0.8] coordinates { (-\offset,\offset) (-\offset-\belly,\width) (-\length,0) (-\offset-\belly,-\width) (-\offset,-\offset)};
		
		\draw [fill=\rotorcolor, fill opacity=\diskopacity] (0,0) circle (\radius);
    }}