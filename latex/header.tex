% Headerdatei

% Dokumentklasse wählen
\documentclass[a4paper,bibliography=totoc,11pt,draft]{scrartcl}
\usepackage{a4wide}
%\usepackage{a4}

% Übersetzungspaket
\usepackage{ngermanb}

% Zeichencodierung
\usepackage[utf8]{inputenc}

% Zeichensatz
\usepackage[T1]{fontenc}
\usepackage{lmodern}

% Anführungszeichen
\usepackage[ngerman]{babel}
\usepackage[ngerman]{translator}  
\usepackage[babel,german=quotes]{csquotes}

% Silbentrennung
\usepackage{hyphenat}

% Mathematik
\usepackage{mathtools}
\usepackage{multirow}
\usepackage{xfrac}

% Einheiten
\usepackage[mediumspace,mediumqspace,squaren]{SIunits}

% Farben
\usepackage{xcolor}
\definecolor{darkgreen}{rgb}{0.0,0.5,0.0}

% Grafiken
\usepackage{graphicx}
\usepackage{float} % lädt das Paket zur Verwendung von zusätzlichen Positionsbefehlen
\usepackage{tikz} % http://cremeronline.com/LaTeX/minimaltikz.pdf
\usepackage{tikz-3dplot} % http://www.texample.net/tikz/examples/the-3dplot-package/
\usepackage{subcaption}
\usepackage{esk} % http://www.bitlib.net/mirror/ctan/macros/latex/contrib/esk/esk.pdf
\usetikzlibrary{decorations.pathreplacing}
\usetikzlibrary{decorations.markings}
\usetikzlibrary{calc}
\usetikzlibrary{shapes,arrows}
\usetikzlibrary{automata}
\usepackage{rotating}

\tikzstyle{block} = [draw, fill=blue!20, rectangle, minimum height=2em]%, minimum width=4em]
\tikzstyle{sum} = [draw, fill=blue!20, circle]%, node distance=1cm]
\tikzstyle{input} = [coordinate]
\tikzstyle{output} = [coordinate]
\tikzstyle{pinstyle} = [pin edge={to-,thin,black}]

% pstricks - Grafiken
\usepackage{pstricks}
\usepackage{pst-plot}
\usepackage{pstricks-add}

% Code Listings
\usepackage{scrhack} 
\usepackage{listingsutf8}

% Code Listings: Formatierung
% ftp://ftp.tex.ac.uk/tex-archive/macros/latex/contrib/listings/listings.pdf
\lstset{
	language=C,
	captionpos=b,
	tabsize=2,
	frame=lines,
	keywordstyle=\color{blue}\bfseries\ttfamily,
	commentstyle=\color{darkgreen},
	stringstyle=\color{red},
	numbers=left,
	numberstyle=\tiny,
	numbersep=5pt,
	breaklines=true,
	showstringspaces=false,
	basicstyle=\footnotesize\ttfamily,
	%showspaces=true,
	%showtabs=true,
	%tab=\rightarrowfill,
	escapeinside={@}{@},
	emph={label},
	%extendedchars=\true,
  inputencoding=utf8
}
\lstset{
  literate=%
					 {ö}{{\"o}}1
					 {ö}{{\"O}}1
           {ä}{{\"a}}1
					 {Ä}{{\"A}}1
           {ü}{{\"u}}1
					 {Ü}{{\"U}}1
   				 {ß}{{\ss}}1
					 {²}{{\textsuperscript{2}}}1
					 {³}{{\textsuperscript{3}}}1
					 {°}{{\degree}}1
}


% Absatz einrücken abstellen (wird in englischsprachigen Dokumenten gemacht)
\setlength{\parindent}{0pt}

% Index
% http://en.wikibooks.org/wiki/LaTeX/Indexing#Sophisticated_Indexing
\usepackage{makeidx}
\makeindex

% Glossar
% ftp://ftp.rrzn.uni-hannover.de/pub/mirror/tex-archive/macros/latex/contrib/glossary/glossary.pdf
% ftp://ftp.tu-chemnitz.de/pub/tex/macros/latex/contrib/glossaries/glossaries-user.html
\usepackage[acronym,toc,numberedsection,numberline,section=section]{glossaries}

% TODO
% http://ftp.fernuni-hagen.de/ftp-dir/pub/mirrors/www.ctan.org/macros/latex/contrib/todo/todo.pdf
\usepackage{todo}

% Klickbare Querverweise
% http://www.math.uni-hamburg.de/home/iffland/Materialien/Einf_hyperref.pdf
%\usepackage[hypertex,bookmarksopen]{hyperref}
\usepackage[bookmarksopen]{hyperref}

% Tabellen
\usepackage{array}
\usepackage{booktabs}


\newcolumntype{C}[1]{>{\centering}m{#1}}

%\usepackage[section] {placeins}

% EOF